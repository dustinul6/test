\documentclass[11pt]{article}
\usepackage{amssymb}
\usepackage{amsfonts}
\usepackage{amsmath}
\usepackage{mathabx}
\usepackage[nohead]{geometry}
\usepackage[singlespacing]{setspace}
\usepackage[bottom]{footmisc}
\usepackage{indentfirst}
\usepackage{endnotes}
\usepackage{graphicx}
\usepackage{rotating}
\usepackage{accents}
\usepackage{natbib}
\newcommand{\ubar}[1]{\underaccent{\bar}{#1}}


\setcounter{MaxMatrixCols}{30}
\newtheorem{theorem}{Theorem}
\newtheorem{acknowledgement}{Acknowledgement}
\newtheorem{algorithm}[theorem]{Algorithm}
\newtheorem{axiom}[theorem]{Axiom}
\newtheorem{case}[theorem]{Case}
\newtheorem{claim}[theorem]{Claim}
\newtheorem{conclusion}[theorem]{Conclusion}
\newtheorem{condition}[theorem]{Condition}
\newtheorem{conjecture}[theorem]{Conjecture}
\newtheorem{corollary}[theorem]{Corollary}
\newtheorem{criterion}[theorem]{Criterion}
\newtheorem{definition}[theorem]{Definition}
\newtheorem{example}[theorem]{Example}
\newtheorem{exercise}[theorem]{Exercise}
\newtheorem{lemma}[theorem]{Lemma}
\newtheorem{notation}[theorem]{Notation}
\newtheorem{problem}[theorem]{Problem}
\newtheorem{proposition}{Proposition}
\newtheorem{remark}[theorem]{Remark}
\newtheorem{solution}[theorem]{Solution}
\newtheorem{summary}[theorem]{Summary}
\newenvironment{proof}[1][Proof]{\noindent\textbf{#1.} }{\ \rule{0.5em}{0.5em}}
\newcommand{\pd}[2]{\frac{\partial#1}{\partial#2}}
\makeatletter
\def\@biblabel#1{\hspace*{-\labelsep}}
\makeatother
\geometry{left=1in,right=1in,top=1.00in,bottom=1.0in}
\DeclareMathOperator*{\argmin}{arg\,min}
\DeclareMathOperator*{\argmax}{arg\,max}

\begin{document}

\title{Purchasing Point Mechanism in Assemblying Complements}

\author{Tzu-Yao Lin
  %\thanks{Address: 3101A Tydings Hall, University of Maryland,
  %College Park, MD 20742, USA, e-mail:
  %\textit{lint@econ.umd.edu}. The author is grateful to XXXX for advice
  %and suggestions, and to XXX, XXX, XXX, and seminar participants at the XXXXXXXX for helpful comments.}
\medskip\\{\normalsize Department of Economics, University of Maryland}}
\maketitle

\sloppy%avoids the breakage of words at the end of lines, by adjusting spaces between words inside the lines

  %\onehalfspacing

  %\begin{abstract}

  %\end{abstract}

\strut

  %\textbf{Keywords:} 

  %\strut

  %\textbf{JEL Classification Numbers:} R14, H0, XY.

  %\pagebreak%breaks to the next page
\doublespacing %makes space between lines to be double, use singlespacing for space 1


\section{Introduction}
 It can be very valuable in some markets for a single buyer to assembly multiple perfect complementary properties. Examples can be seen in urban rendevelpment projects, large public constructions and patent pools industries. For those projects, there is one single buyer that needs the consent of every individual property owner's permission in order to carry out his project. This has been a very difficult task in real world and in market design literature. n private market is a difficult task. 

Problem of Market Design/Mechanism Design
Property Rights/Individual Rationality
Connection to public good problem




   \bibliographystyle{abbrv}
    \bibliography{bibtex_proposal.bib}
    \end{document}


